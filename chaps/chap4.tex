\section{Checkpointing}
\label{chap:Checkpointing}

\subsection{Wiederholung}
Beim Vorwärtsmodus ($F'(x)\cdot s$) und Rückwärtsmodus ($s^TF'(x)$) hatten wir gesehen, dass für $F:\mathbb{R}^n\mapsto\mathbb{R}^n$ folgende Abschätzungen gelten.

\begin{tabular}{L L}
	OPS(EVAL(F'(x)\cdot s)) &\le c_1 \cdot OPS()Eval(F(x))\\
	OPS(EVAL(s^T\cdot F'(x))) &\le c_2 \cdot OPS()Eval(F(x))
\end{tabular}\\
und somit

\begin{tabular}{L}
	OPS(EVAL(F'(x))) \le MIN(n,m) OPS(EVAL(F(x))
\end{tabular}\\
insbesondere

\begin{tabular}{L}
	OPS(EVAL(vf(x))) \le c OPS(EVAL(f(x)))
\end{tabular}\\
für $f:\mathbb{R}^n\mapsto \mathbb{R}^n$ im Gegensatz zu FD, wo $n+1$ Auswertungen von $f$ benötigt werden.\\

\noindent
\underline{Wichtig:}
Die Abschätzungen gelten nur für die Anzahl der Operationen!!!

\noindent Rechenoperationen sind meist billig im Vergleich zu Lese-/Schreibzugriffen auf fast allen Systemen.

\subsection{Forward/ Reverse Mode und Speicherbedarf}

Forward Mode $\dot{y} = F'(x)\dot{x}$ : 
\begin{tabular}{L L L }
	v_{i-n} & = x_i& i = 1\dots n\\
	\dot{v}_{i-n} &= \dot{x}_i&\\
	\hline
	v_i &=\varphi (v_j)_{j\prec i}&i=1\dots l\\
	\dot{v}_i &= \sum_{j\prec i} \frac{\partial}{\partial v_j} \varphi_i<(u_i)\dot{v}_j&\\
	\hline
	y_{m-i} &=v_{l-i}&i= (m-1) \dots 0\\
	\dot{y}_{m-i} &= \dot{v}_{l-i}&\\
	\dot{y} &= \varPhi \ \ \ \ \ \ \varPhi \dot{x}&
\end{tabular}
$\rightarrow \dot{v}$ wird nur so lange benötigt, wie $v$ benötigt wird, kann also während der Laufzeit schon gelöscht werden. Mit anderen Worten der Speicheraufwand für den Forward-Mode entspricht in etwa dem Speicheraufwand der Funktionsauswertung (bis auf konstanten Faktor).

$\rightarrow$ Forward Mode stellt kein Problem dar\\
anders sieht es im Backward Mode aus:\\
$\bar{x}=\bar{y}F'(x)$
\begin{tabular}{L L}
	v_{i-n}=\dot{x}_i & i = 1-n,\dots,l\\
	v_i = \varphi( )  & i= 1, \dots, l\\
	y_{m-i} = v_{l-i} & i=m-1, \dots, 0\\
	\hline
	\bar{v}_{l-i}= \bar{y}_{m-i} & i = 0,\dots, m-1\\
	\bar{v}_i = \sum_{i\succ j}\bar{v}_i\frac{\partial}{\partial v_j} \varphi_i(u_i) & i = l,\dots,1\\
	\bar{x}_i = \bar{v}_{i-n} & i=n,\dots,1
\end{tabular}
$$\bar{x}=\bar{y}\ \underrightarrow{\varPhi(\ ) \dots \varPhi(\ )}$$


Hier müssen/müssten alle Zwischenwerte $v_i$ gespeichert werden, insbesondere kann der Speicherbedarf nun für generelle Funktionen nicht mehr abgeschätzt werden.

\subsubsection{Typisches Beispiel}
Beispiel: Berechne $\nabla f(u_i(x))$, wobei $u_i(x)$ die Lösung einer gewöhnlichen DGL (ODE)
$$\frac{\partial u}{\partial t} = F(u,x,t)\ \ \ \forall_{t \in T} = [t_0,\dots,t_l]$$
ist, welche von unbekannter Variable x abhängt, z.B. Parametern oder Startwerte.

\subsubsection{Explizites SIR Bsp}
 Die ODE für ein simples SIR ist gegeben durch:
$$\frac{\partial u}{\partial t}
\begin{Bmatrix}
\frac{\partial S}{\partial t} = & -\alpha S \cdot I & = F_1(u,x,t)\\
\frac{\partial I}{\partial t} = & +\alpha S \cdot I - \beta I &= F_2(u,x,t)\\
\frac{\partial R}{\partial t} = & + \beta I & = F_3(u,x,t)
\end{Bmatrix}
= F(u,x,t)
$$
wobei die Übergangswerte $\alpha$ und $\beta$ sowie die Startwerte $(S_0,I_0,R_0)$ für $U=(S,I,R)$ oft unbekannt sind, d.h. $x=(\alpha,\beta,S_0,I_0,R_0)$. Numerisch kann die Lösung für $U=(S,I,R)$ durch das explorative Eulerverfahren approximiert werden über einem diskretisierten Zeitintervall $T = [t_0, \dots, t_N]$.\\
\begin{tabular}{L}
	S_0 = x_3\\
	I_0 = x_4\\
	R_0 = x_5\\
\end{tabular}\\

\begin{algorithmic}
	\For{$k=1 \dots N$}
		\State $U_k \gets T_k-T_{k-1}$
		\State $S_k \gets S_{k-1} + U_kF_1(S_{k-1},I_{k-1},R_{k-1},x,t)$
		\State $I_k \gets I_{k-1} + U_kF_2(S_{k-1},I_{k-1},R_{k-1},x,t)$
		\State $R_k \gets R_{k-1} + U_kF_3(S_{k-1},I_{k-1},R_{k-1},x,t)$
	\EndFor
	\State \Return $f(S_0,\dots S_N,I_0,\dots I_N, R_0,\dots R_N)$
\end{algorithmic}

Hier könnnte z.B. $f$ eine numerische Approximation 
$$f(S_0,\dots,S_N,I_0,\dots I_N,\mathbb{R}_0,\dots,\mathbb{R}_N)$$
$$ \frac{1}{T_N-t_0}\sum_{k=1}^N h_k(\frac{I_k+I_{k-1}}{2})$$
für die durchschnittliche (Rel.) Anzahl von Infizierten sein:
$$\frac{1}{T}\int_TI(t)dt$$
oder der fehlenden gemessenen Daten $d_{k}$, $k \in 0,\dots N$
$$f(u,x,t) = \frac{1}{N}\sum_{k=1}^N(d_k-I_k)^2$$
$\Rightarrow$ Gradient zeigt Sensitivität der Modellparameter bezgl. Störung an und/oder kann zur Optimierung benutzt werden (Data-Fitting).

\noindent\makebox[\linewidth]{\rule{\paperwidth}{0.4pt}}
Viele dieser und ähnlicher Anwendungen (typischerweise numerische Approximation von ODE's/ PDE's) haben wie im vorherigen Beispiel oft die folgende Form:
$$v_0 \rightarrow F(v_0) = v_1 \rightarrow F(v_1)=v_2 \rightarrow \dots F(v_{N-1})=v_N$$
bzw.
\begin{algorithmic}
	\State INT $v_0$
	\For{$k=1 \dots N$}
		\State $v_k \gets F(v_{k-1})$
	\EndFor
	\State \Return $v_n$
\end{algorithmic}

D.h. alle Zwischenvariablen $v_k$ müssen gespeichert werden.

Alternative: \glqq Speicheraufwand gegen Rechenaufwand austauschen \grqq\\
m.a.w. anstatt alle $v_k$ tz speichern, werden nur ein paar ausgewählte Zwischenwerte gespeichert, sogenannte Checkpoints, und die restlichen benötigten Werte werden nach Bedarf ausgehend von den CP neu errechnet.

\begin{tabular}{L L | L}
	&v_0 &\\
	&\rightarrow F(v_0) = v_1 &v_1 \hat{=}CP_1\\
	&\rightarrow F(v_1)=v_2 &\\
	(*)&\rightarrow \dots &\\
	&\rightarrow F(v_i)=v_{i+1} &v_{i+1}\hat{=}CP_2\\
	&\rightarrow F(v_{i+1})=v_{i+2} &\\
	&\rightarrow F(v_{N-2})=v_{N-1}&v_{N-1}\hat{=}CP_3\\
	&\rightarrow \dots &\\
	&\rightarrow F(v_{N-1})=v_N& v_N\hat{=}CP_4
\end{tabular}
\newpage
\subsubsection{Reverse Mode (für obiges Beispiel)}
\begin{itemize}
	\item[1.] Auswerten von $(*)$ und speichern der CP
	\item[2.] Reverse Sweep
	\begin{itemize}
		\item[i)]  mittels $CP_4$ ausrechnen von $\bar{y}_{N-1}=\bar{y}_N^T F_{N-1}'(v_N)$
		\item[ii)] mittels $CP_3$ und $\bar{Y}_{N-1}$ausrechnen von $\bar{y}_{N-2}=\bar{y}_{N-1}^T F_{N-2}'(v_{N-1})$
		\item[iii)]
		\begin{itemize}
			\item Neuberechnung von  $v_{n-2}$ durch Forward Sweep ausgehend von $CP_2$, dann Auswerten von $\bar{y}_{N-3}=\bar{y}_{N-2}^T F_{N-3}'(v_{N-2})$
			\item Neuberechnung von  $v_{n-3}$ durch Forward Sweep ausgehend von $CP_2$, dann Auswerten von $\bar{y}_{N-4}=\bar{y}_{N-3}^T F_{N-4}'(v_{N-3})$
			\item $\vdots$
			\item Neuberechnung von  $v_{i+2}$ durch Forward Sweep ausgehend von $CP_2$, dann Auswerten von $\bar{y}_{i+1}=\bar{y}_{i+2}^T F_{i+1}'(v_{i+2})$
		\end{itemize}
		\item[iv)] Auswertung von $\bar{y}_{i}=\bar{y}_{i+1}^T F_{i}'(v_{i+1})$
		\item[v)]
		\begin{itemize}
			\item Neuberechnung von $v_i$ durch Forward Sweep Ausgehend von $CP_1$, dann Auswerten von $\bar{y}_{i-1}=\bar{y}_{i}^T F_{i-1}'(v_{i})$
			\item $\vdots$
		\end{itemize}
		\item[vi)] $\vdots$
		\item[vii)] Auswerten von $\bar{x}=\bar{y}_0=\bar{y}_1^TF_0'(v_1)$
	\end{itemize}
\end{itemize}

Beispielgrafik (Berechnungsreihenfolge mit Pfeilen visualisiert)\\
\vspace{3cm}

Beobachtung:
\begin{itemize}
	\item je mehr CP's desto weniger Werte müssen neu berechnet werden
	\item Extremfall \glqq alle Zwischenwerte sind CP\grqq\ liefert ursprüngliche Methode
	\item je weniger CP's desto mehr neuberechnete Werte, dafür weniger teure Speicheroperationen
\end{itemize}

Typische Perfomanzkurve (Grafik: Verhältnis Runtime zu log(#CP's))\\
\vspace{3cm}

$\Rightarrow$ mögliche Laufzeit für Vanilla RM ohne CP kann z.T. schneller/langsamer oder etwas dazwischen sein im Vgl. zu mit CP - abhängig von Kosten des Speicherzugriffs\\

\noindent weitere Beobachtung:
\begin{itemize}
	\item manche CP's werdem ab einem bestimmten Zeitpunkt nicht mehr benötigt $\rightarrow$ können wiederverwendet werden um andere zw.zuspeichern
\end{itemize}
Frage:\\
- Welche Werte sollen als CP gespeichert werden?\\
- Wie sieht das aus, wenn CP wieder benutzt werden?

\subsection{Binomial Checkpointing}
Idee:\\
Verteile CP's optimal bzgl. des Gesamtaufwandes, d.h. benutze auch CP's die wieder frei werden
$\rightarrow$ CP's nicht uniform verteilen\\
Grafik\\
sondern binomial\\
Grafik\\

Vorteil:\\
die Letzten CP werden früh frei und können wieder verwendet werden und Neuberechnungen für $(*)$ werden eingespart\\
$\Rightarrow$ frei gewordene CP's können dann beim ersten Durchlauf von $(*)$ wieder benutzt um Zwischenwerte abzuspeichern.

Beispiel: Revolle Paper \dots / Walther







\noindent\makebox[\linewidth]{\rule{\paperwidth}{0.4pt}}
Zusätzliche evtl. doppelte oder unvollständige Notizen zum Kapitel aus der Vorlesung:\\

Wdh. Checkpointing\\
Gegeben $F = F_v \odot F_{v-1} \odot \dots \odot F_1$ D.H.
$x_0 = x \rightarrow x_1 = F_1(x_0) \rightarrow x_2 = F_2(x_1) \rightarrow \dots \rightarrow y= x_v = F_v(x_{v-1})$

Problem: bei Rückwärtssweep müssen alle Werte $x_i$ vorhanden sein (gespeichert werden)

$\bar{x}= \bar{y}^TF'(x) | \bar{x} = \bar{y}_1F'(x_0)  \leftarrow \dots \leftarrow $

Lösung: Checkpoints (CP) - anstatt aller $x_i$  werden nur ausgewählte Werte $x_i$ gespeichert, fehlende werden neuberechnet basierend auf diesen CP

Bsp:

\begin{tabular}{ L L L}
	\text{Startwert} & \text{CP-werte} & \text{Zielwert}\\
	0 \rightarrow & 1 \rightarrow 2 \rightarrow \dots \rightarrow 9 \rightarrow & 10 \\	
\end{tabular}

Vorgehen: Init 0


\begin{tabular} { c L L L }
	Eval + Save & & & \\
	CP : & 0 \rightarrow & 1 \rightarrow 2 \rightarrow \dots \rightarrow 9 \rightarrow & 10 \\
	& \downarrow & \dots \downarrow \dots \downarrow \dots & \\
	Eval + Rev & 9 \rightleftarrows 10 & & \\
	&  \downarrow & & \\
	Load CP+ Eval & 7 \rightarrow 8 & &\\
	Eval + Rev & 8 \rightarrow 9 & &\\
	&  \downarrow & & \\
	Load CP+ Eval & 7 \rightleftarrows 8 & &\\
	
	letzte beiden Load CP + (Eval) + Rev & & &\\
\end{tabular}

Frage: wie werden die Checkpoints verteilt?
$\rightarrow$ Faustregel: kann als dynamisches Optimierungsproblem definiert werden \glqq$MIN t(L,C)$\grqq, wobei 
$$t(l,C) = MIN_{1\leq \tilde{l}\leq} \{\tilde{l} + t( l- \tilde{l},c-1)+t(\tilde{l},c)\}$$
mit
$ t(l,c) \approx$ Zeit für RV von $l$ Teilfkt mit C Checkpoints\\

\vspace{1cm}

Grafik\\

\noindent
Optimierungsproblem hat explizite Form \marginpar{nicht Prüfungsrelevant}
$$t(l,c) = v(l,c) \cdot l - \beta (c+1, r-1)$$
wobei $\beta(c,r) = \frac{(c+v)!}{c!r!}$ und $r = r(l,c)$
die eindeutige natürliche Zahl mit $\beta (c,r-1) < l \leq \beta (c,r)$

(r $\approx$ Ganzzahliges Verhältnis von Kosten für Reversal von l Funktionen mit c Checkpoints zu den Kosten für eine Vorwärtsauswertung)

Prop 12.3 \glqq Bin. Rev. Schedule \grqq\\
die optimalen CP sind gegeben durch 
$$ l(c,r) = \beta(c,r) = \binom{c+v}{c} = \frac{(c+r)!}{c!r!}$$

Faustregel:
\begin{itemize}
	
	\item CP ungefähr bei Hälfte des verbliebenen Abschnitts setzen
	
	Grafik
	\item wenn viele CP vorhanden (im Vgl. zu Rel. Auswertungsaufwand), dann CP eher link von der Mitte
	
	Grafik
	\item ansonsten eher rechts
	
	Grafik
\end{itemize}
