\section{Methoden zur Berechnung von Ableitungen ($f: \mathbb{R}^n \mapsto \mathbb{R}$)}


\subsection{numerisch / Differenzenquotient}
\label{subsec:numerisch / Differenzenquotient}
Idee: approximiere den Grenzwert
\begin{align*}
\frac{\partial f}{\partial x_i} & = lim_{h\mapsto 0} \frac{f(x_1,\dots,x_n) - f(x_i,\dots,x_n)}{h}\\
 &= lim_{h\mapsto 0} \frac{f(x+h \cdot l_i) - f(x)}{h}
\end{align*}
wobei $l_i$ an der $i$-ten Stelle den Wert 1 und sonst den Wert 0 hat, durch 
$$\frac{\partial f}{\partial x_i}(x) \approx \left[\frac{f(x+e,\tilde h)-f(x)}{\tilde h)}\right]$$
mit "hinreichend kleinem"\ Wert $\tilde h \in R^+$
\\
Vorteil:
\begin{itemize}
	\item einfach zu implementieren
	\item benötigt nur Auswertung für f
\end{itemize}
Nachteil:
\begin{itemize}
	\item $\tilde h$ ist unbekannt, "falsche"\ Wahl von $\tilde h$ kann zu Rundungs-/ Approximationsfehlern führen
	\item Berechnung von $\nabla f$ benötigt $n+1$ Auswertungen der Funktion $f$ - teuer für eine große Anzahl von Variablen
\end{itemize}
Grafik zur Genauigkeit: Fehler V -förmig (Rundungsfehler und Ungenauigkeit)

\vspace{3cm}

\subsection{Analytisches Differenzieren/ per Hand}
Idee: Benutze Standard-Differentationsregeln zum Ableiten einer Funktion, d.h. Kettenregel / Produktregel / Divisionsregel / Summenregel / Winkelfunktionen / Weitere

\noindent Vorteil:
\begin{itemize}
	\item kann sehr effiziente, genaue Ergebnisse liefern
\end{itemize}
Nachteil:
\begin{itemize}
	\item fehleranfällig
	\item z.T. sehr aufwendig, wenn überhaupt möglich
\end{itemize}
Beispielrechnung
$$f:\mathbb{R}^2\mapsto\mathbb{R}, f(x_1,x_2) = \frac{x_2 x_1^2}{sin(x_1)}+cos(x_2)$$

%\begin{align*}
%	\frac{\partial f(x_1,x_2)}{\partial x_1} &= \frac{\partial }{\partial x_1}
%	 = \frac{2x_2x_1}{den}\\	
%\end{align*}
\vspace{3cm}
\subsection{Symbolisches Differenzieren}
Idee: Zerlegung einer Funktion in ihre elementaren Operationen als Berechnungsbaum und anschließende Transformation mittels vorheriger Ableitungsregeln darstellen
Beispiel: $f$ wie zuvor, Grafik: symbolischer Berechnungsbaum von $f$

\vspace{8cm}

wird zu, Grafik:  symbolischer Berechnungsbaum mit 41 Knoten

\vspace{10cm}

Vorteil:
\begin{itemize}
	\item exakte Ableitungen
	\item automatisierbar
	\item nachvollziehbar
	\item (kann effizient sein)
\end{itemize}
Nachteil
\begin{itemize}
	\item speicherintensiv!!
\end{itemize}
typisches Gegenbeispiel (Speelpenning):
$f:\mathbb{R}^n\mapsto\mathbb{R}, f(x) = \prod_{i=1}^n x_i$
Bemerkung: $\prod_{i=1}^n x_i$ taucht z.B. in der Berechnung von Determinanten o.ä. linear Algebra OPs vor.

\subsection{Automatisches/ Algorithmisches Differenzieren}

gegeben: eine mathematische Funktion: $F:\mathbb{R}^n\mapsto\mathbb{R}^m$, welche
\begin{itemize}
	\item hinreichend oft differenzierbar ist
	\item durch einen endlichen "Straight-Line" Code beschrieben werden kann, d.h. keine (unendliche) Rekursion und Verzweigungen (if-branches,\dots) enthält
\end{itemize}
Mit anderen Worten $F:U\subseteq \mathbb{R}^n\mapsto\mathbb{R}^m, F(x) = y$ ist gegeben durch eine sogenannte Evaluation Procedure
\begin{align*}
	v_{i-n}= x_i & i = 1,\dots,n && Eingabe\\
	v_i = \ \varphi_i(v_j) & i = 1,\dots,l && Auswertung\\
	y_{m-i}& i = m-1,\dots,0 && Ausgabe
\end{align*}
wobei $x \in \mathbb{R}$ die unabhängigegen, $y \in \mathbb{R}^m$ abhängigen, $v_i \in \mathbb{R}$ Zwischenvariablen und 
$$\varphi \in \Phi = \{\pm, cos, sin, \sqrt{\cdot}, Exp, Log \dots \}$$
hinreichend oft differenzierbare Elementarfkt. (über U) sind.

\vspace{\baselineskip}
\noindent Ziel: 

effiziente numerische Bestimmung von Ableitungsinformationen der Funktion bezüglich ihrer Ausgabe $y=F(x)$ und Eingabe $x \in \mathbb{R}^n$
\vspace{\baselineskip}

\noindent Bemerkung:

Die Zwischenvariablen $v_i$ in der Eval. Proc. können nach ihrer Berechnungsvorschrift sortiert werden und induzieren einen "Computational DAG", in welchem jede Zwischenvariable durch einen Knoten repräsentiert wird. Eine Kante existiert wenn eine solche Variable direkt von einer Anderen abhängt.
$v_i = \varphi(v_j)_{j \prec i}$ ist eine Kurzschreibweise für den Funktionswert der entsprechenden Elementarfkt. $\varphi_i$ zur Auswertung von $v_i$, die von den vorher berechneten Werten ($(v_j)_{j \prec i}$) direkt abhängt.

\vspace{\baselineskip}

\noindent Forward Mode

Ausgehend von der Evaluation Procedure kann man eine erweiterte/ extended  Eval. Proc. angeben, welche sowohl $y=F(x)$ als auch $\dot{y}=F'(x)\dot{x}$ für $\dot{x}\in \mathbb{R}$ berechnet.

\begin{tabular}{| L | L | L |}
	\hline
	v_{i-n} = x_i & i=1,\dots ,n& Eingabe\\
	\dot{v}_{i-n} = \dot{x}_i & &\\
	\hline
	v_i = \ \varphi_i(v_j) & i = 1,\dots,l & Auswertung\\ 
	\dot{v}_i = \sum_{j \prec i} \frac{\partial}{\partial v_j}\varphi_i(u_i)\cdot \dot{v}_j & u_i = (v_j)_{j\prec i} &\\
	\hline
	y_{m-i}& i = m-1,\dots,0 & Ausgabe\\
	\dot{y}_{m-i}& &\\
	\hline
\end{tabular}
\newpage
\noindent Beispiel : $f : \mathbb{R}^2\mapsto\mathbb{R}$ wie zuvor gegeben\\

\begin{tabular}{| L | L |}
	\hline
	v_{-1} = x_1& \\
	\dot{v}_{-1} = \dot{x}_1& Eingabe\\
	v_0 = x_2&\\
	\dot{v}_0 = \dot{x}_2&\\
	\hline
	v_1 = v_{-1}^2&\\
	\dot{v}_1 = 2 v_{-1} \cdot \dot{v}_{-1} &\\
	v_2 = v_0 v_1 &\\
	\dot{v}_2 = \dot{v}_0v_1 + v_0\dot{v}_1&\\
	v_3 = sin(v_{-1})&\\
	\dot{v}_3 = cos(v_{-1})\dot{v}_{-1} & Auswertung\\
	v_4 = v_2 / v_3&\\
	\dot{v}_4 = 1\cdot \frac{\dot{v}_2}{v_3^3} + \frac{-v_2}{v_3^2}\cdot \dot{v}_3&\\
	v_5 = cos(v_0)&\\
	\dot{v}_5 = -sin(v_0)\dot{v}_0&\\
	v_6 = v_4 + v_5&\\
	\dot{v}_6 = 1\cdot \dot{v}_4 + 1 \cdot \dot{v}_5&\\
	\hline
	y_1 = v_6&Ausgabe\\
	\dot{y}_1 = \dot{v}_6&\\
	\hline
\end{tabular}
\quad\quad
\begin{tabular} { L }
	V = (0, \dots, 0)\\
	\quad\quad\downarrow \Phi_{-1}\\
	V = (v_{-1},0, \dots, 0)\\
	\quad\quad\downarrow \Phi_0\\
	V = (v_{-1}, v_0,0, \dots)\\
	\quad\quad\downarrow \Phi_1\\
	V = (v_{-1}, v_0, v_{-1}^2,0, \dots)\\
	\quad\quad\downarrow \Phi_2\\
	V = (v_{-1}, v_0, v_{-1}^2, v_{-1}v_0,0, \dots)\\
	\quad\quad\scalebox{1}[2]{\rotatebox{-90}{$\dashrightarrow$}}\\
	\\
	V = (v_{-1}, v_0, \dots, v_4+v_5, 0)\\
	\quad\quad\downarrow \Phi_7\\
	V = (v_{-1}, v_0, \dots, v_4+v_5, v_6)\\
\end{tabular}

\vspace{\baselineskip}

\noindent
\begin{tabular}{L | P{1.2cm}  L}
	\hline
	\text{Grafik: Computational Graph für }f \quad\quad\quad\quad& Tabelle&\text{: part.Ordnung der }v_i\text{'s}\\
	&$\varnothing$ &\prec -1\\
	&$\varnothing$ &\prec 0\\
	&$-1$ &\prec 1\\
	&$0,1$ &\prec 2\\
	&$-1$ &\prec 3\\
	&$2,3$ &\prec 4\\
	&$0$ &\prec 5\\
	&$4,5$ &\prec 6\\
\end{tabular}

