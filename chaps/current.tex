%\begin{document}

\section{currentstuff}

Wdhlg.: Ziel effiziente Berechnung von dünnbesetzten Abhleitungen $F'(x)$, deren dünnbesetzte Struktur bekannt ist.

$\rightarrow$ Ausnutzen FM und RM, d.h. finde Seedmatrix (Richtungen)
$S = S_1, \dots , S_p \in \mathbb{R}^{n\times p}$ Richtungen \\
so dass die Matrix $B=F'(x) \cdot S$ alle Informationen von $F'(x)$ enthält.\\

\noindent
\underline{'Curtis-Powell-Reid Seeding'}\\
Mittels Coloring/ Färbung des Spaltenzusammenhangsgraphen $G=(V,E)$ von $F'(x)$ bestehend aus Knoten $V=\{1dots n\}$, die jeweils eine Spalte von $F'(x)$ entsprechen und Kanten $E\subseteq V\times V$, die durch wiederholte auftretende Einträge in den Spalten induziert werden, mit anderen Worten es existiert eine Kante $(i,j)$ zwischen Knoten $i$ und $j$ gdw. Spalte $i$ und $j$ von $F'(x)$ mindestens einen gemeinsamen Nulleintrag besitzen.\\

\noindent
Beispiel: $*$ -Nichtnulleintrag, $0$ - Null\\
$\begin{bmatrix}
*	& *	& 0	& 0	\\
*	& 0	& *	& 0	\\
*	& 0	& 0	& *	\\
0	& 0	& *	& *	\\
\end{bmatrix}
$\\
Spaltenzusammenhangsgraph:\\
$V=\{1,2,3,4\}$\\
$E=\{(1,2),(1,3),(1,4),(3,4)\}$\\
Zeilenzusammenhangsgraph:\\
$V=\{1,2,3,4\}$\\
$E=\{(1,2),(2,3),(2,4),(3,4)\}$\\

Übung
$\begin{bmatrix}
 *	& *	& *	& *	& 0	& 0	& 0	& 0	& 0	& 0\\
 *	& *	& 0	& 0	& *	& *	& 0	& 0	& 0	& 0\\
 *	& *	& 0	& 0	& 0	& 0	& *	& *	& 0	& 0\\
 0	& 0	& *	& *	& *	& *	& 0	& 0	& 0	& 0\\
 0	& 0	& *	& *	& 0	& 0	& * & *	& 0	& 0\\
 0	& 0	& *	& *	& 0	& 0	& 0 & 0	& *	& *\\
\end{bmatrix}$\\

Färbung des Spaltenzshgsgraphen\\
$C:\{1\dots n\}\mapsto \{1\dots p\}$\\
mit (möglichst wenig) $p$ Farben (1-Rot, 2-Grün, 3-Blau, 4-Gelb,\dots)

liefert dann die entsprechende Seedmatrix
$$S = [e^T_{c(j)}]_{j=1,\dots,n}\in \mathbb{R}^{n\times p}\ \ \ e_k^T = [0 \dots 0,1_k, 0 \dots 0]$$\\
und Anzahl der notwendigen Richtungen $p$ (d.h. nötige Aufrufe des FM). Hier bezeichnet  $e_k$ den $k$-ten kartesischen Basisvektor.\\

\noindent
Grafik:\\3 mal Spaltenzusammenhangsgraph nebeneinander, verschiedene Färbungsvarianten (4Farben (1: Rot(1), ) bzw. 3 (2 und 3) Farben)

$ S^I = \begin{bmatrix}
1	& 0	& 0	& 0	\\
0	& 1	& 0	& 0	\\
0	& 0	& 0	& 1	\\
0	& 0	& 1	& 0	\\
\end{bmatrix}_{4\times 4}$

$ S^{II} = \begin{bmatrix}
1	& 0	& 0	\\
0	& 1	& 0	\\
0	& 1	& 0	\\
0	& 0	& 1	\\
\end{bmatrix}_{3\times 4}$


$ S^{III} = \begin{bmatrix}
1	& 0	& 0	\\
0	& 1	& 0	\\
0	& 0	& 1	\\
0	& 1	& 0	\\
\end{bmatrix}_{3\times 4}$\\

Beispielmatrix\\
$\begin{bmatrix}
*	& *	& 0	& 0	\\
*	& 0	& *	& 0	\\
*	& 0	& 0	& *	\\
0	& 0	& *	& *	\\
\end{bmatrix}
=
\begin{bmatrix}
a_{11}	& a_{12}	& 0	& 0	\\
a_{21}	& 0	& a_{23}	& 0	\\
a_{31}	& 0	& 0	& a_{34}	\\
0	& 0	& a_{43}	& a_{44}	\\
\end{bmatrix}
$\\

$AS^I =
\begin{bmatrix}
 a_{11}	& a_{12}& 0		& 0 \\
 a_{21}	& 0		& 0		& a_{23}\\
 a_{31}	& 0		& a_{34}& 0\\
 0		& 0		& a_{44}& a_{43}\\
\end{bmatrix}$
$AS^{II}=
\begin{bmatrix}
a_{11}	& a_{12}& 0		& 0 \\
a_{21}	& 0		& 0		& a_{23}\\
a_{31}	& 0		& a_{34}& 0\\
0		& 0		& a_{44}& a_{43}\\
\end{bmatrix} INCORRECT$
$AS^{III}=,
\begin{bmatrix}
a_{11}	& a_{12}& 0		& 0 \\
a_{21}	& 0		& 0		& a_{23}\\
a_{31}	& 0		& a_{34}& 0\\
0		& 0		& a_{44}& a_{43}\\
\end{bmatrix}INCORRECT$\\

\noindent
Problem: Rekonstruktion von $A (=F'(x))$ aus $B=A\cdot S$\\
$\Rightarrow$ Lsg von $b_i^T = w_i^TS_i$ bzgl. $w_i^T$ liefert die $p_i$ Nichtnulleinträge $\nabla F_i(x)=w_i^T$der $i$-ten Zeile\\
Beispielgrafik der Matrixberechnung\\
\noindent

$\Rightarrow\ b_i^T =w_i^TS_i = e_i A S = e_i \sum_{j \in X_i} Ae_je_j^TS$\\
wobei  $X_i \approx$ Sparsitypattern der $i$-ten Zeile von $F'(x)$ ist, d.h. die Indizes der N.N. von $v\cdot F_i'(x)$\\
$\Rightarrow\ b_i^T =w_i^TS_i \Leftrightarrow S_i^Tw_i = b_i$ ist ein überstimmtes LGS!\\

$\Rightarrow$ Löse stattdessen $\tilde{S}_i^Tw_i = \tilde{b}_i$ mit $\tilde{S}^T_i \in \mathbb{R}^{p_i\times p_i}$ mit $S_i^T = \begin{Bmatrix}
S_i\\
\dots \\
S_i \\
\end{Bmatrix}_{p_i \times p_i}$\\
CONTENT MISSING


Im Falle von CPR (Curtis-Powell-Reid) ist\\
$$S_i = \left[l^T_{c(j)}\right]_{j\in X_i}\in \mathbb{R}^{p_i\times p}$$
gegeben durch permutierte Einheitsmatrix mit angehängter Nullmatrix ist.


Beispiel zur Matrix $A\begin{bmatrix}
a_{11}	& a_{12}	& 0	& 0	\\
a_{21}	& 0	& a_{23}	& 0	\\
a_{31}	& 0	& 0	& a_{34}	\\
0	& 0	& a_{43}	& a_{44}	\\
\end{bmatrix}$:
Sparsity-Pattern: für $i$ Zeilen,jeweils die Indexmenge der Nichtnulleinträge:\\
\begin{tabular}{L}
	x_1 = \{1,2\}\\
	x_2 = \{1,3\}\\
	x_3 = \{1,4\}\\
	x_4 = \{3,4\}\\
\end{tabular}
$S^{III} = \begin{bmatrix}
1	& 0	& 0	\\
0	& 1	& 0	\\
0	& 0	& 1	\\
0	& 1	& 0	\\
\end{bmatrix}
$\\
$S_1^{III} = \begin{bmatrix}
1	& 0	& 0	\\
0	& 1	& 0	\\
\end{bmatrix}$
$S_2^{III} = \begin{bmatrix}
1	& 0	& 0	\\
0	& 0	& 1	\\
\end{bmatrix}$
$S_3^{III} = \begin{bmatrix}
1	& 0	& 0	\\
0	& 1	& 0	\\
\end{bmatrix}$
$S_4^{III} = \begin{bmatrix}
0	& 0	& 1	\\
0	& 1	& 0	\\
\end{bmatrix}$\\
$b_1 = [a_{11}, a_{12},0]$
$b_2 = [a_{21}, 0, a_{23}]$
$b_3 = [a_{31}, a_{34},0]$
$b_4 = [0,a_{44}, a_{43},0]$\\

$\tilde{S}_1^{III} = \begin{bmatrix}
1&0\\
0&1\\
\end{bmatrix}$
$\tilde{S}_2^{III} = \begin{bmatrix}
1&0\\
0&1\\
\end{bmatrix}$
$\tilde{S}_3^{III} = \begin{bmatrix}
1&0\\
0&1\\
\end{bmatrix}$
$\tilde{S}_4^{III} = \begin{bmatrix}
0&1\\
1&0\\
\end{bmatrix}$\\
$\tilde{b}_1 = [a_{11},a_{}]$
$\tilde{b}_2 = [a_{},a_{}]$
$\tilde{b}_3 = [a_{},a_{}]$
$\tilde{b}_4 = [a_{},a_{}]$\\

\noindent\underline{Zeilenkompression}\\
Analog zur Spaltenkompression von Rechts mittesls des FM ($F'(x)\cdot S)$ kann auch von links komprimiert werden mit dem Zeilen
zshgsgraph und mittels des RM ($S\cdot F'(x)$),
Mit anderen Worten betrachte nun
$$ C^T = W^T F'(x)$$
wobei $W^T \in \mathbb{R}^{q\times m}$ nun die adjugierte Richtungen und $c^T \in \mathbb{R}^{q \times n}$ dei komprimierte Darstellung der Jacobimatrix ist.\\

\noindent
Zeilen- und Spaltenkompression gleichzeitig machen:\\
typisches Gegenbeispiel

$\begin{bmatrix}
* & \dots & * \\
\vdots & \ddots & 0\\
*	&	0	& *\\
\end{bmatrix}$
hier funktioniert nur eines der beiden nicht