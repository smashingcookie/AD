\section{Lineare Gleichungslöser}
Gegeben:\\
Matrix $ A \subseteq \mathbb{R}^{n\times n}, b \in \mathbb{R}^n$\\
Ziel:\\
Finde $w\in \mathbb{R}^n$ mit $Aw=b$\\
Bemerkung:\\
Sei Funktion $F:\mathbb{R}^{n\times n} \times \mathbb{R}^n \mapsto \mathbb{R}^n$,die eine Lösung von berechnet $w=F(A,b)\in \mathbb{R}^n = A^{-1}b$
welche z.B. als  Methode/Code $w=foo=(A,b)$gegeben ist.

Bsp.: In Matlab kann der "\" Operatore benutzt werden
Function w = FOooo(A,b)
$\dots$



Die meisten linearen Gleichungslöser benutzen/basieren auf iterativen Methoden, wie z.B. dem konjugierten Gradienten Verfahren,
(falls $A=A^T$ symmetrisch ist und $A>0$ [ist lösbar])\\

\subsection{CG-Verfahren}
Literatur: Wiki, Nocedal, Num. Optimization\\
(wird nicht abgefragt)
\begin{algorithmic}
	\State $A,b,w_0$
	\State $w_0$
	\State $v_0 \gets b-Aw_0$
	\State $d_0 \gets v_0$
	\For{$k=0,1,\dots$}
		\State $\vdots$
	\EndFor
\end{algorithmic}
\begin{itemize}
	\item[$\Rightarrow$] $(*)$t A und nicht A selbst $\rightarrow$ es müssen nur ein paar Vektoren der Dimension $n$ gespeichert werden und keine $n\times n$ Matrix (Matrixfreies Verfahren)
	\item[$\Rightarrow$] Konvergenz des Verfahrens hängt hauptsächlich von der Eigenwertstruktur (Eigenwerte $\lambda$: $Av = \lambda v,\ \ , \lambda \in \mathbb{R}$) der Matrix $A$ ab
\end{itemize}

Diagramme zum Konvergenzverhalten\\
\vspace{3cm}

\noindent
\begin{itemize}
	\item[$\Rightarrow$] meist sehr effizient, (evtl. Preconditioning nötig)
	\item[$\Rightarrow$] Insbesondere ist CG aufgrund von $(*)$ bei der Schrittberechnung nach Newton 
	$$F'(x)S = -F(x)$$
	zum lösen von $F(x)=0$ für $F:\mathbb{R}^n\mapsto\mathbb{R}^n$, d.h. $A=F'(x)$, $b=-F(x)$ von Vorteil
	\begin{itemize}
		\item[$\Rightarrow$] $As$ = F'(x) kann via FM für CG-Verfahren ...ausgerechnet werden
	\end{itemize} 
\end{itemize}

\noindent
\subsubsection{Problem}
Annahme $w = A^{-1}b = F(A,b)$ ist nun selber Teil eines Codes/Simulation $h$ die abgeleitet werden soll, und $A$ sowie $b$ hängen von einem Parameter $x$ ab. D.h. $w(x)= A^{-1}(x)b(x)$ und gesucht
$$\frac{\partial w(x)}{\partial x} = \frac{\partial h}{\partial x}\cdot \frac{\partial w}{\partial x}$$
Das Benötigt $\frac{\partial w}{\partial x}$ bzw. $\frac{\partial F(A(x),b(x)}{\partial x}$.\\
$\Rightarrow$ Diese Ableitung ist typischerweise numerisch instabil und ineffizient
\marginpar{nicht mehr glatt, If-Verzweigungen $\leftrightarrow$ Straight-Line Code}
Lösung: Gray-Box-Ansatz, d.h. anstatt $\frac{\partial h(x)}{\partial x}$ einfach komplett mittels AD zu berechnen, bestimme Teile der Ableitung per Hand und klebe diese anschließend wieder zusammen
$$\frac{\partial w(x)}{\partial x} = \frac{\partial}{\partial} \frac{\partial}{\partial}$$
ADD CONTENT

Lösung:\\
Ersetzen von Teilableitungen des Codes durch geeignete analytische Lösungen für $Aw = b$\\
Literatur dazu: Homepage: Tapenade, Inria: Linear Solver
\marginpar{Aufgabe: Herleitung lesen, auf Homepage statt $\bar{b}$ $bb$, \dots}

FM : $w = Foo(A,b)$\\
$[w, \dot{w}]=Foo_D(A,\dot{A},b,\dot{b})$\\
$w = A\setminus b$\\
$\dot{w}=A\setminus (\dot{b}-\dot{A}w)$
End\\

RM: $[w, \bar{b},\bar{A}] = Foo_B[A,b,\bar{w}]$\\
$w= A\setminus b$\\
$\bar{b}+ =A^-T\bar{w}= A' \setminus \bar{w}$\\
$\bar{A}_{i,j}-= w_j\bar{b}_i$\\
$\bar{W}=0$\\
end
