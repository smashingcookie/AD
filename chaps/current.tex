\section{currentstuff}

Jacobi-Matrix-Berechnung mittels Graphreduktion

\begin{itemize}
	\item[Ziel:] Bestimme $F'(x)\in\mathbb{R}^{m\times n}$ für eine allgemeine differenzierbare Funktion $F:\mathbb{R}^n\mapsto\mathbb{R}^m$, $y=F(x)$ via Reduktion des Berechnungsgraph
	\item[Bsp.:] $$F(x_1,x_2) = [(x_1+x_2)cos((x_1+x_2)^2),x_2 + cos((x_1+x_2)^2)]$$
	$$v_{-1} = x_1,\ v_0 = x_2,\ v_1 = v{-1}+v_0,\ v_2=v_1^2,\ v_3=cos(v_2)$$
\end{itemize}
$(*)$ Berechnungsgraph: (Grafik)\\
\vspace{2cm}

\noindent
$(\tilde{*})$ Berechnungsgraph für die erweiterte Auswertungsprozedur: (Grafik)\\
$(**)$ $v_i = \sum_{j\prec i}\frac{\partial}{\partial}\varphi_i(u_i)\cdot \dot{v}_j\ ,\ \ \ c_{i,j}=\frac{\partial}{\partial v_j}\varphi_i(u_i)$):\\
\vspace{2cm}

mit $c_{1,-1}=1$, $c_{1,0}=1$, $c_{2,1}=2v_1$, $c_{3,2}=sin(v_2)$, $c_{4,1}=v_3$, $c_{4,3}=v_1$, $c_{5,3}=1$, $c_{5,0}=1$.

\noindent
$(\hat{*})$ Angenommen der Graph hätte nicht diese Form, sondern wäre bipartit, dann könnte man $F'(x)$ einfach ablesen: (Grafik)\\
\vspace{3cm}

\noindent
$$F'(x) =
\begin{bmatrix}
	\nabla F_1(x)\\
	\vdots\\
	\nabla F_m(x)
\end{bmatrix}
\text{ für } [F(x) = F_1(x), F_2(x), \dots, F_m(x)]^T$$
ADD STUFF HERE

\begin{itemize}
	\item[]Frage:] Kann $(\tilde{x})$ auf die Form $(\hat{*})$ gebracht werden, mittels geeigneter Knoten- oder Kantenmodifikationen ohne die zugrunde liegende Ableitungsfunktion zu verändern.
	\item[Antwort:] \underline{JA}, mittels sog. \underline{Vertex}, \underline{Edge} und Faceelimination.
	
\end{itemize}
\noindent
Beispiel fortgeführt:\\
Offensichtlich sind die folgenden Graphen Äquivalent zu $(\tilde{*})$\\
(Grafik [$(\tilde{*})$ ohne Kante $c_{4,3}$])\\
\vspace{2cm}

\noindent
mit $\tilde{c}_{4,1}= c_{4,3}\cdot c_{3,2} \cdot c_{2,1} + c_{4,1}$
(Grafik)\\

\vspace{2cm}
mit $\tilde{c}_{4,-1} = \tilde{c}_{4,1}\cdot c_{1,-1}$ und $\tilde{c}_{4,0}=\tilde{c}_{4,1}\cdot c_{1,0}$

Bsp fortgeführt:\\
$\Rightarrow$ erinnert schon starl an $(\hat{*})$ und kann weiter fortgesetzt werden, bis der Graph bipartit ist. Diese Methode heißt Kantenelimination. Anstatt Kanten können auch Knoten eliminiert werden (Vertex).\\
$\vdots$\\
Grafiken zur Knotenelimination+Berechnungen\\
$\vdots$

\subsection{Eliminationsregeln}
Die zugrunde liegende Ableitungsfunktion wird erhalten bei der Entfernung einer Kante $(i,j)$, d.h. $c_{i,j}=0$ mittels Rückwärts Kantenelimination, welche die Kantengewichte wie folgt anpasst:
$$c_{i,k} += c_{i,j}\cdot c_{j,k} \text{ für alle } j\prec i$$
bzw.
$$c_{h,j} += c_{h,i}\cdot c_{i,j} \text{ für alle } h \succ i $$
im Falle der Vorwärtskanntenelimination (und dann $c_{i,j} =0$).\\
Im Falle der Knotenelimination müssen die Kantengewichte für alle Paare $i\succ j\ ,\ k\succ i$ mit 
$$ c_{i,k}+=c_ij\cdot c_jk$$
inkrementiert werden, bevor die angrenzenden Kanten zum Knoten $j$ null gesetzt werden, bzw. $j$ gelöscht werden kann.\\
(Grafik Kantenelimination vorwärts $|$ rückwärts):\\

\vspace{4cm}