%\begin{document}
	
\section{currentstuff}

ForwardMode: $J \cdot v$ : Jacobi * Vektor\\
BackwardMode: $v^T \cdot J$: Vektor$^T$ * Jacobi

Ziel: effiziente Auswertung von $v^TF'(x) = \dot{y}$
Überlegung: $\dot{y} = P \Phi'_l(V_l)\cdot \Phi'_{l-1}(V_{l-1})\dots \Phi'_1(V_1)\cdot Q \cdot \dot{x}$

wie im Forward Mode zu berechnen muss nun

$\dot{x} = \bar{y}^TP\Phi'_l(V_l)\dots \Phi'_1(V_1)\cdot Q$
ausgewertet werden

Non-Incremental:

\begin{tabular}{| L L |}
	v_{i-n} = x &i=1\dots n\\
	v_i=\varphi_i(v_j)_{j\prec i} &i=1 \dots l\\
	y_{m-i} = v_{l-i} & i=m-1 \dots 0\\
	\hline
	\bar{y}^TP\bar{v}_{l-i} = \bar{y}_{m-l}& i =m-1 \dots 0\\
	\bar{v}^T\Phi'(\dots)\bar{v}_j = \sum_{i\succ j} \bar{v}_j&\\
	&\\
\end{tabular}

Incremental: (Zusatzmaterial)

\begin{tabular}{| L L |}
	v_{i-n} = x &i=1\dots n\\
	v_i=\varphi_i(v_j)_{j\prec i} &i=1 \dots l\\
	y_{m-i} = v_{l-i} & i=m-1 \dots 0\\
	\hline
	\bar{v}_i = 0 & i =1-n \dots l\\
	\bar{v}_{l-1} = &\\
	\bar{v}^T\Phi'(\dots)\bar{v}_j = \sum_{i\succ j} \bar{v}_j&\\
	&\\
\end{tabular}


$\rightarrow$ Auswertung von $\bar{x} = \bar{y}^TF'(x)$ benötigt eine Auswertung von F (Forward-Sweep) und einen Reverse Sweep. Der Aufwand kann nach oben abgeschätzt werden

$$OPS(EVAO(\bar{y}^TF'(x))) \leq c_2 OPS(EVAL(F(k)))$$,
wobei $C_2$ eine Konstante ist $(c_2\approx 2-5)$
\\
Fehler bisher, sollte sein: (inkorrekte Korrektur, unvollständig)
$$\bar{v}_i \sum_{i\succ j} \bar{v}_i \frac{\partial}{\partial v_j}\varphi_i(u_i)$$


Bsp. Reverse Sweep (Non-Incremental) $f(x_1,x_2)$
wir wollen berechnen $\bar{y}^Tf'(x), f: \mathbb{R}^2\mapsto\mathbb{R}^1$

\begin{tabular}{| L  L |}
	\bar{y}_1 = 1 &\\
	\bar{v}_6 = \bar{y}_1 &\\
	\bar{v}_5 = \bar{v}_6 \frac{\partial}{\partial v_5}(v_4+v_5) & = \bar{v}\cdot 1\\
	\bar{v}_4 = \bar{v}_6 \frac{\partial}{\partial v_4}(v_4+v_5)& = \bar{v}\cdot 1\\
	\bar{v}_3 = \bar{v}_4 \frac{\partial}{\partial v_3}(\frac{v_2}{v_3})&  = \bar{v}_4 \cdot \left( -\frac{v_2}{v_3^2}\right)\\
	\bar{v}_2 = \bar{v}_4 \frac{\partial}{\partial v_2}(\frac{v_2}{v_3})&) =\bar{v}_4 \cdot \left(\frac{1}{v_3}\right)\\
	\bar{v}_1 = \bar{v}_2 \frac{\partial}{\partial v_1}(v_0 \cdot v_1) & =\bar{v}_2 \cdot v_0\\
	\bar{v}_0 = \bar{v}_2 \frac{\partial}{\partial v_0}(v_0 \cdot v_1) + \bar{v}_5 \frac{\partial}{\partial v_0}(cos(v_0))& =  =\bar{v}_2 \cdot v_1 + \bar{v}_5(-sin(v_0))\\
	\bar{v}_{-1} = \bar{v}_2 \frac{\partial}{\partial v_0}(v_0 \cdot v_1) + \bar{v}_5 \frac{\partial}{\partial v_0}(cos(v_0))& =2 \bar{v}_1 v_{-1} + \bar{v}_3 cos(v_{-1})\\
%	\hline
\end{tabular}

\vspace{\baselineskip}
Aufgaben:
\begin{itemize}
	\item Fehler von \ref{subsec:numerisch / Differenzenquotient} nachimplementieren
	\item Forward Mode implementieren
	\item Rosenbrock
\end{itemize}

Datentyp adouble erstellen, der 2 Werte speichert: einen für den Wert selbst und eines für die Richtung -> wichtiger Teil Fkt. a la operator+ 

in Matlab: Adimat (Uni Darmstadt)
%\end{document}